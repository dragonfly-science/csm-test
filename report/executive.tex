\section*{Executive Summary}

The expansion of Vibrant Planet's platform for natural resource management requires high-quality data to guide forestry management in the state of California. Best practice decision-making is currently limited to areas where LiDAR datasets are available. Within this report we provide a proof-of-concept how we aim extrapolate point cloud derived canopy height information to the whole state of California.

 The machine-learning trains Sentinel-2 satellite data to upscale canopy height information beyond the scope of current field data collections. This task is performed using our own workflows to generate canopy height models with improved quality terrain surfaces (25.1 \%) from unclassified LiDAR data; and due to the high quality of the Sentinel-2 imagery the procedure operates at unprecedented resolution (10 m) for an upscaling operation at this scale.  While the prototype in this early stage has no refined expectations towards accuracy, it demonstrates that the highly technical task of ingesting, aligning and analysing large datasets can be conducted in an efficient, replicable and scalable process. With the infrastructure backbone of our model in place, additional spectral bands, vegetation indices, LiDAR point clouds and other state-wide layers (e.g. forest classification) can be ingested readily into future models; expecting to vastly improve the quality of our predictions.

This documents gives an overview of the state of this project, a brief breakdown of the methodological framework for the prototype and a literature research - revealing the extent of research considered to be deployed. A detailed documentation for geospatial processes, machine-learning concepts and data processing can be found in the respective repositories.

\emph{TODO: A summary of the models developed, and performance of them}
