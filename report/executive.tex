\section*{Executive Summary}

The expansion of Vibrant Planet's platform for natural resource management requires high-quality data to guide forestry management in the State of California. Using a fine-scale land-cover model, the Canopy Height Model (CHM), best-practice decision-making is currently limited to areas where LiDAR (light detection and ranging) datasets are available.
To address this limitation, the current project explored approaches and datasets to develop a CHM test case using publicly-available data for a test region
in California.  The exploration included the consecutive upscaling of LiDAR-derived canopy height information from local study areas to the whole of California. Our aim is to upscale using Sentinel-2 imagery. This report documents the current
 proof of concept, including geospatial processes and machine-learning. It also outlines its potential expansion for extrapolating point cloud derived canopy height information for the entire State of California.
%Detailed documentation of geospatial processes, machine-learning concepts and data ingestions via Amazon Web Services (AWS) that were part of this project
%approaches for This project provides a proof-of-concept how we aim extrapolate point cloud derived canopy height information to the whole state of California.

 The machine-learning concept trained Sentinel-2 satellite data to upscale canopy height information beyond the
 scope of current field data collections. This task was performed using processes developed here to generate
 canopy height models with improved quality terrain surfaces (25.1\% more ground points incorporated than existing layers) from unclassified LiDAR data. Owing to the high quality
 of the Sentinel-2 imagery, this is an unprecedented resolution (10m) for an upscaling operation at this scale.
Although the prototype in this early proof of concept had no refined expectations towards accuracy, it demonstrated that the technical
task of ingesting, aligning and analysing extensive datasets can be conducted in an efficient, replicable and scalable process.

With the infrastructure framework of this model in place, additional spectral bands, vegetation indices, LiDAR point clouds and other state-wide layers
(e.g; forest classification) can be readily ingested into future models,. This addition of data and expansion of the model is
expected to greatly improve the quality of the predictions.

%This documents gives an overview of the state of this project, a brief breakdown of the methodological framework for the prototype and a literature research - revealing the extent of research considered to be deployed. A detailed documentation for geospatial processes, machine-learning concepts and data processing can be found in the respective repositories.
%
%\emph{TODO: A summary of the models developed, and performance of them}
